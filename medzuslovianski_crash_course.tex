\documentclass{article}
\usepackage[utf8]{inputenc}
\usepackage{multicol}
\usepackage[margin=1in]{geometry}
\usepackage{titlesec}
\usepackage[utf8]{inputenc}
\usepackage{pdflscape}
\usepackage[a4paper, margin=1cm]{geometry}
\usepackage{tabularx}
\usepackage{rotating}
\usepackage{array}
\usepackage[table]{xcolor}
\usepackage{caption}
\titleformat{\section}{\Large\bfseries}{\thesection}{1em}{}
\title{Interslavic crash course for slavs}
\begin{document}
\maketitle
\newpage
\section{Forword}
This little course has been created to help speakers of Slavic languages learn to write and speak Interslavic in a natural way. The course is intentionally short, as Slavs already passively understand most of the grammar and vocabulary, and the goal here is to activate this knowledge. By working through the word lists and inflecting words with the grammar tables, learners can start building sentences.

I have carefully selected words for the word list that are widely understood by most Slavs. For cases where South Slavs might not be familiar with a word, I have provided an alternative. 
\\ \\
At the end of this book, you will find 50 example sentences to inspire you to create your own Interslavic sentences. I recommend printing the word lists and grammar tables on separate sheets and arranging them side by side on your desk. This will make the writing process more efficient.
\\ \\
By the time you reach the end of the 50 example sentences, you’ll find it easy to create your own sentences. Feel encouraged to contribute additional words, improved grammatical explanations, or even a cover design for this course. Your contributions will, of course, be credited. My goal is to turn this crash course into a community project because I believe that more Slavic speakers should discover the beauty of Interslavic and form connections with each other and for this easily accessible learning material is needed. The book will be open source on my github.

The grammar tables and the numbered section were created based on Jan van Steenbergen’s grammar articles, available at steen.free.fr. I hope he is comfortable with this usage.

Good luck with your journey in learning Interslavic/Medžuslovjanski! \\
-Aleksander Ginschel 2025
\section{Helpful links}
\begin{itemize}
    \item https://github.com/ginschel/interslavic\_crash\_course/tree/master (in case you want to contribute)
    \item https://interslavic-dictionary.com (in case you don't know a word or want to compare languages)
    \item https://interslavic.fun/learn/grammar (good side for learning grammar)
    \item http://steen.free.fr/interslavic/grammar.html (the original web site with the grammar articles from interslavic.fun)
    \item https://hackmd.io/@xlO6E2n-S5eX4tROzsbTpg/ryr9565Ni (this is a german-english-interslavic phrase book with practical sentences)
    \item https://www.youtube.com/@interslavicofficial (if you want to hear very well spoken interslavic)
    \item https://linktr.ee/interslavicmedzuslovjansky
\end{itemize}
\newpage
\section{Words}
\section*{osnova/fundamental words}

\begin{multicols}{3}
\begin{itemize}
 \item ako
\item ale / no
\item až (ili daže ili jug. čak)
\item bez
\item bliz
\item bo (north slavic, jer south slavic)
\item bolje
\item čemu ili od kakogo povoda
\item često
\item črěz
\item čto
\item da by
\item daže
\item dlja (south slavic zaradi)
\item do
\item dokud
\item iz
\item iznova
\item jedino
\item ješče
\item k
\item kako
\item kako jesi
\item koliko
\item kogda (ch. kada)
\item kto
\item ktory / south slavic koj
\item li (vměsto či)
\item medžu
\item menje
\item město
\item mnogo (vměsto bardzo ili duže)
\item može
\item možno
\item na
\item na priměr
\item nad
\item najbolje
\item najdti
\item najmenje ili hot by
\item nazad
\item nema
\item nečesto ("časom" ne vsegda razumlivo)
\item nehaj
\item neželi
\item něčto 
\item něhaky
\item několik
\item něktory
\item ni... ni...
\item ničto
\item nikto
\item o
\item ob
\item očevidno
\item od
\item od povoda
\item odkud
\item oprosti / pardon
\item po
\item pod
\item poněkogda 
\item potom
\item potomu že
\item pozdrav
\item prěd
\item pri
\item s
\item samo
\item směsta
\item snova
\item sějčas
\item sovsěm
\item sovsěm
\item tako
\item takože
\item tamtoj ili toj
\item tamo
\item togda
\item toliko - tak dužo
\item trěba (ili potrěbno)
\item tu
\item tutoj
\item tutčas
\item tutdenj
\item uže
\item vsaky
\item vsaky (vměsto každy. to dlja jug ne razumlivo)
\item vsegda
\item za
\item zaisto (vměsto napravdu)
\item začto
\item zato
\item zdravo (do vitanja) ili dobrodošli
\item že

\end{itemize}
\end{multicols}

\newpage
\section*{imeslova/nouns}

\begin{multicols}{3}
\begin{itemize}
 \item avto
\item besěda
\item butilka ili fljaška
\item čas
\item časina
\item čaša
\item čest
\item člověk
\item denj
\item dělo
\item děte
\item dožd
\item dolžnost
\item dolžina
\item dom
\item država
\item dveri
\item glava
\item glas
\item god
\item grad
\item groši
\item hlěb
\item hvala ili jug blagodarjenje
\item ime
\item izbor
\item jeda ili jelo (jug)
\item kava
\item kniga
\item kompjuter
\item konec
\item lice
\item maslo
\item meso
\item minuta
\item mlěko
\item muž
\item měsec ili luna 
\item noč
\item noga
\item oběd
\item oko
\item okno
\item oružje
\item osnova
\item otec
\item pivo
\item podloga ili tlo (jug)
\item pokoj
\item pogled
\item pogoda
\item pogrěška
\item polovina
\item pomoč
\item povod
\item pravda ili jug. istina
\item prědmet
\item prijatelj
\item problem
\item pytanje
\item raz
\item restoran
\item robota
\item ruka
\item ryba
\item sala
\item sedmica ili tydenj (jug)
\item slovo
\item slovjanin
\item slučaj
\item sněg
\item sok
\item solnce
\item stol
\item strana
\item svět
\item uho
\item večerja
\item večinstvo
\item vino
\item voda
\item vojna
\item vrěme
\item včera
\item vkus ukus
\item zadanje
\item zamok
\item zautra
\item zautraka
\item zdravje
\item žena
\item život
\item značenje
\item zemja
\end{itemize}
\end{multicols}
bonus: ponedělok, vtorok, srěda, četvrtok, petok, subota, nedělja

\newpage
\section*{adjectives}

\begin{multicols}{2}
\begin{itemize}
 \item běly
\item bystry ili brzy (jug)
\item cěly
\item dobry
\item dolgy
\item drugy
\item hladny
\item kratky
\item krasivy / krasny
\item legky
\item maly
\item nebystry 
\item nebrzy (jug)
\item novy
\item poslědnji
\item poslědny
\item potrěbny
\item prijatny
\item prvy
\item prosty
\item razumlivy
\item sivy
\item slědny
\item složeny 
\item stary
\item taky
\item te
\item teply
\item težky
\item ugodny
\item umorjeny
\item uvěrjeny = isty
\item važny
\item veliky
\item vid
\item vkusny
\item zly
\item črny

\end{itemize}
\end{multicols}

\small
\setlength{\tabcolsep}{2pt}
\renewcommand{\arraystretch}{1.1}

% Define custom colors
\definecolor{myyellow}{RGB}{255, 255, 153}
\definecolor{mygreen}{RGB}{118, 253, 163}
\definecolor{myblue}{RGB}{23, 226, 240}
\definecolor{myorange}{RGB}{243, 163, 26}
\definecolor{myothergreen}{RGB}{69, 206, 10}

\noindent
\begin{sidewaystable}
\section*{glagoly/verbs}
\begin{tabularx}{\linewidth}{|*{9}{>{\raggedright\arraybackslash}X|}}
\hline
\cellcolor{myyellow}irreg. -ut & 
\cellcolor{myyellow}nuti -ne- , ut & 
\cellcolor{myyellow}idti -ide- ut & 
\cellcolor{myyellow}reg ě jut & 
\cellcolor{mygreen}reg a jut & 
\cellcolor{myothergreen}ovati -uje- jut & 
\cellcolor{myblue}i konj et & 
\cellcolor{myblue}hoditi & 
\cellcolor{myblue}irreg -et \\
\hline
brati -bere- & stanuti -stane- & prijdti -prijde- & razuměti & dělati & egzistovati & ljubiti & odhoditi & stojati -stoji- \\
\hline
vzeti -vazme- & dvignuti -dvigne- & najdti -najde- & uměti & imati & potrěbovati & (ob)govoriti & vhoditi & sěděti -sědi- \\
\hline
dati -da(d)- & dosegnuti -dosegne- & vyjdti -vyjde- & & pytati & probovati & mysliti & uhoditi & spati -spi- \\
\hline
dostati -dostane- & tegnuti -tegne- & izojdti -izojde- & & znati & funkcjonovati & nositi & prihoditi & (u)viděti -vidi- \\
\hline
htěti -hče- & & & & nazyvati & & odgovoriti & vyhoditi & běgti -běži- \\
\hline
mogti -može- (gut) & & & & padati & & otvoriti & & gleděti -gledi- \\
\hline
nesti -nese- & & & & skakati & & prositi & & \\
\hline
ostati -ostane- & & & & pozdravjati & & značiti & & \\
\hline
biti -bije- & & & & rabotati & & jezditi -jezdi- & & \\
\hline
piti -pije- & & & & zautrakati & & učiti & & \\
\hline
početi -počne- & & & & posluživati se & & pozdraviti & & \\
\hline
prinesti -prinese- & & & & zapametati & & umoriti se & & \\
\hline
stati -stane- & & & & želati & & zabezpametiti & & \\
\hline
(s)kazati -skaže- skaži! & & & & igrati & & voditi & & \\
\hline
žiti -žive- & & & & čitati & & variti & & \\
\hline
jesti -je- (dut) & & & & plavati & & postaviti & & \\
\hline
iskati -ište- & & & & kuhati & & hvaliti & & \\
\hline
(na)pisati -piše- & & & & slušati & & (s)tvoriti & & \\
\hline
nadějati se -naděje- (jut) & & & & posluživati se & & voziti & & \\
\hline
vezti -veze- & & & & koristati & & blagodariti & & \\
\hline
vesti -vedi- & & & & & & završiti & & \\
\hline
čuti -čuje- & & & & & & & & \\
\hline
klasti -klade- & & & & & & & & \\
\hline
zvati -zove- & & & & & & & & \\
\hline
pekti -peče- (kut) & & & & & & & & \\
\hline
\end{tabularx}
\\
\\
The columns are colored by their conjugation and their endings. The conjugation pattern is always -m, -š, -Ø, -mo, -te, -(j)ut/-et.
\end{sidewaystable}
\newpage
\section*{Frazy/phrases}
\begin{itemize}
    \item to jest vyše prosto
    \item to byla dobra besěda
    \item želam dobrogo vkusa/ukusa/apetita
    \item probovati najdti (for some slavs easier to understand than iskati and infamous szukati)
    \item Ščestnogo dnja rodženja!
\end{itemize}
\section*{Extra Verbs/Glagoly}
\begin{itemize}
    \item dějati se -děje-
    \item alt. to dostati -> (jug) dobiti/polučiti 
\end{itemize}

\section*{Numbers}
0-10: 0. nula, 1. jedin (jedna, jedno), 2. dva (dvě), 3. tri, 4. četyri, 5. pet, 6. šest, 7. sedm, 8. osm, 9. devet, 10. deset.
\\ \\
(11-19) are formed by adding -nadset (pronounced -nacet) to the numbers 1-9: 11. jedinnadset, 12. dvanadset, 13. trinadset, 14. četyrinadset, 15. petnadset, 16. šestnadset, 17. sedmnadset, 18. osmnadset, 19. devetnadset.
\\ \\
(20-90) are formed by adding -deset to the numbers 2-9: 20. dvadeset, 30. trideset, 40. četyrideset, 50. petdeset, 60. šestdeset, 70. sedmdeset, 80. osmdeset, 90. devetdeset.
\\ \\
(100-900) are formed by adding -sto to the numbers 2-9: 100. sto, 200. dvasto, 300. tristo, 400. četyristo, 500. petsto, 600. šeststo, 700. sedmsto, 800. osmsto, 900. devetsto
\\ \\
Alternatively, the hundreds can also be formed by inflecting the word sto, resulting in the following set: 100. sto, 200. dvěstě, 300. trista, 400. četyrista, 500. petsot, 600. šestsot, 700. sedmsot, 800. osmsot, 900. devetsot.
\\ \\
The words for thousand, million and milliard are: tyseč (1000), milion and miliard. 
\\ \\
for instance: 5421 = petsot četyristo dvadeset jedin
\section{Grammar tables}
Every inflection that is in brackets will be used if the ending of the noun stem is a soft letter. 
\section*{Nouns}
\begin{table}[h!]
\centering
\renewcommand{\arraystretch}{1.3}
\begin{tabular}{|l|c|c|c|c|c|}
\hline
\textbf{} & \multicolumn{3}{c|}{\textbf{I-Declination}} & \textbf{II} & \textbf{III} \\
\hline
\textbf{} & \textbf{m. (anim.)} & \textbf{m. (inanim.)} & \textbf{n.} & \multicolumn{2}{c|}{\textbf{f.}} \\
\hline
\textbf{nom.} & \texttt{-i} & \texttt{-y (-e)} & \texttt{-a} & \texttt{-y (-e)} & \texttt{-i} \\
\hline
\textbf{acc.} & \texttt{-ov (-ev)} & \texttt{-y (-e)} & \texttt{-a} & \texttt{-y (-e)} & \texttt{-i} \\
\hline
\textbf{gen.} & \multicolumn{2}{c|}{\texttt{-ov (-ev)}} & \texttt{-Ø} & \texttt{-Ø} & \texttt{-ij} \\
\hline
\textbf{dat.} & \multicolumn{5}{c|}{\texttt{-am}} \\
\hline
\textbf{ins.} & \multicolumn{5}{c|}{\texttt{-ami}} \\
\hline
\textbf{loc.} & \multicolumn{5}{c|}{\texttt{-ah}} \\
\hline
\end{tabular}
\caption{The third declination includes all feminine nouns ending with -ost like kost. }
\end{table}

These words have some irregular forms, but apart from that they are declinated like words form the third declination.
\begin{itemize}
    \item člověk (m.), pl. ljudi
    \item děte (gen. děteta or dětete) (n.), pl.děti
    \item oko (n.), pl. oči
    \item uho (n.), pl. uši
\end{itemize}

\section*{Adjectives}
\begin{table}[h!]
\centering
\renewcommand{\arraystretch}{1.3}
\begin{tabular}{|l|c|c|c|c|}
\hline
\textbf{} & \textbf{m. (animate)} & \textbf{m. (inanimate)} & \textbf{n.} & \textbf{f.} \\
\hline
\textbf{nom.} & \multicolumn{2}{c|}{\texttt{-y (-i)}} & \texttt{-o (-e)} & \texttt{-a} \\
\hline
\textbf{acc.} & \texttt{-ogo (-ego)} & \texttt{-y (-i)} & \texttt{-o (-e)} & \texttt{-u} \\
\hline
\textbf{gen.} & \multicolumn{3}{c|}{\texttt{-ogo (-ego)}} & \texttt{-oj (-ej)} \\
\hline
\textbf{dat.} & \multicolumn{3}{c|}{\texttt{-omu (-emu)}} & \texttt{-oj (-ej)} \\
\hline
\textbf{ins.} & \multicolumn{3}{c|}{\texttt{-ym (-im)}} & \texttt{-oju (-eju)} \\
\hline
\textbf{loc.} & \multicolumn{3}{c|}{\texttt{-om (-em)}} & \texttt{-oj (-ej)} \\
\hline
\end{tabular}
\caption{Adjective case endings by gender and animacy}
\end{table}
\section*{Pronouns}
\begin{table}[h!]
\centering
\renewcommand{\arraystretch}{1.3}
\begin{tabular}{|l|c|c|c|c|c|}
\hline
\textbf{} & \textbf{1st person} & \textbf{2nd person} & \textbf{3rd masc.} & \textbf{3rd neut.} & \textbf{3rd fem.} \\
\hline
\textbf{nom.} & \texttt{ja} & \texttt{ty} & \texttt{on} & \texttt{ono} & \texttt{ona} \\
\hline
\textbf{acc.} & \texttt{mene (me)} & \texttt{tebe (te)} & \multicolumn{2}{c|}{\texttt{jego (go)}} & \texttt{ju} \\
\hline
\textbf{gen.} & \texttt{mene} & \texttt{tebe} & \multicolumn{2}{c|}{\texttt{jego}} & \texttt{jej} \\
\hline
\textbf{dat.} & \texttt{mně (mi)} & \texttt{tobě (ti)} & \multicolumn{2}{c|}{\texttt{jemu (mu)}} & \texttt{jej} \\
\hline
\textbf{ins.} & \texttt{mnoju} & \texttt{toboju} & \multicolumn{2}{c|}{\texttt{jim}} & \texttt{jeju} \\
\hline
\textbf{loc.} & \texttt{mně} & \texttt{tobě} & \multicolumn{2}{c|}{\texttt{jim}} & \texttt{jej} \\
\hline
\end{tabular}
\caption{Personal pronouns in singular by case and gender}
\end{table}

\begin{table}[h!]
\centering
\renewcommand{\arraystretch}{1.3}
\begin{tabular}{|l|c|c|c|c|}
\hline
\textbf{} & \textbf{1st person} & \textbf{2nd person} & \textbf{3rd m.anim.} & \textbf{3rd other} \\
\hline
\textbf{nom.} & \texttt{my} & \texttt{vy} & \texttt{oni} & \texttt{one} \\
\hline
\textbf{acc.} & \texttt{nas} & \texttt{vas} & \texttt{jih} & \texttt{je} \\
\hline
\textbf{gen.} & \texttt{nas} & \texttt{vas} & \texttt{jih} & \texttt{jih} \\
\hline
\textbf{dat.} & \texttt{nam} & \texttt{vam} & \multicolumn{2}{c|}{\texttt{jim}} \\
\hline
\textbf{ins.} & \texttt{nami} & \texttt{vami} & \multicolumn{2}{c|}{\texttt{jimi}} \\
\hline
\textbf{loc.} & \texttt{nas} & \texttt{vas} & \multicolumn{2}{c|}{\texttt{jih}} \\
\hline
\end{tabular}
\caption{Plural personal pronouns by case and gender}
\end{table}
\newpage
\begin{table}[htbp]
\centering
\renewcommand{\arraystretch}{1.3}
\begin{tabular}{|l|c|c|c|}
\hline
\textbf{Case} & \textbf{Who?} & \textbf{What?} & \textbf{Reflexive Pronoun} \\
\hline
\textbf{nom.} & \texttt{kto} & \texttt{čto} & \texttt{-Ø} \\
\hline
\textbf{acc.} & \texttt{kogo} & \texttt{čto} & \texttt{sebe (se)} \\
\hline
\textbf{gen.} & \texttt{čego} & \texttt{čego} & \texttt{sebe} \\
\hline
\textbf{dat.} & \texttt{komu} & \texttt{čemu} & \texttt{sobě (si)} \\
\hline
\textbf{ins.} & \texttt{kym} & \texttt{čím} & \texttt{soboju} \\
\hline
\textbf{loc.} & \texttt{kom} & \texttt{čem} & \texttt{sobě} \\
\hline
\end{tabular}
\caption{Pronouns for "Who?", "What?" and Reflexive Pronouns by Case}
\end{table}
\section{Exercices}
\section*{Sentences to translate into medžuslovjanski}
\begin{enumerate}
    \item How are you?  
    \item That is good.  
    \item Why are you here?  
    \item How are you going?  
    \item I like this.  
    \item Are you tired?  
    \item What do you want to do?  
    \item Can I help you?  
    \item I see you.  
    \item Ask me.  
    \item We can go.  
    \item He is going now.  
    \item No problem.  
    \item The last thing I want is this.  
    \item That is really difficult.  
    \item Because I am not sure.  
    \item I like this city.  
    \item That is my friend.  
    \item Give me that.  
    \item You have many questions.  
    \item I don't know how to do this.  
    \item My father works there.  
    \item That is your choice.  
    \item We are drinking coffee.  
    \item We are learning now.  
    \item Are you ready?  
    \item It is very nice here.  
    \item Now we must go.  
    \item What are you doing?  
    \item This is my home.  
    \item She is at the restaurant.  
    \item Health is the most important.  
    \item I greet you.  
    \item We can start.  
    \item I like your book.  
    \item Are you tired?  
    \item We must write.  
    \item When will we eat?  
    \item She likes to drink juice.  
    \item Because it is not there.  
    \item Dinner is ready.  
    \item What is happening?  
    \item Thank you for the help.  
    \item Sometimes I go there.  
    \item That is not very clear.  
    \item Dinner is really tasty.  
    \item It is beautiful here.  
    \item Can you tell me?  
    \item I finished the work.  
    \item When is your birthday?  
\end{enumerate}




\end{document}


